\documentclass[12pt,titlepage]{article}

\usepackage{amssymb,amsmath} % usual symbols
\usepackage{amsthm} % for theorem environments
\usepackage{tikz-cd} % for drawing commutative diagrams

\usepackage{hyperref} % needed for fancy theorem referencing
\hypersetup{colorlinks,citecolor=black,filecolor=black,linkcolor=blue,urlcolor=black}
\usepackage{theoremref} % for fancy theorem referencing

\newtheorem{thm}{Theorem}[section]
\newtheorem{cor}{Corollary}[thm]
\newtheorem{lem}[thm]{Lemma}
\theoremstyle{remark}
\newtheorem{rmk}[thm]{Remark}
\newtheorem*{exm}{Example}

\newcommand{\mb}{\mathbb}
\newcommand{\mf}{\mathbf}
\newcommand{\mc}{\mathcal}
\newcommand{\mfk}{\mathfrak}
\newcommand{\spec}[1]{\text{Spec}\left(#1\right)}
\DeclareMathOperator{\Hom}{Hom}

\title{}
\date{\today}
\author{}
\begin{document}

\section{Factoring algorithm}

We start off by analysing CVP for the lattice and vector

\[\mc L=\begin{pmatrix}c_1&0&\dots&0&0\\0&c_2&\dots&0&0\\\vdots&\vdots&\ddots&\vdots&\vdots\\0&0&\dots&c_{n-1}&0\\B\log p_1&B\log p_2&\dots&B\log p_{n-1}&B\log p_n\end{pmatrix}\quad\mf a=\begin{pmatrix}0\\0\\\vdots\\0\\B\log N\end{pmatrix}\]

and assume that $c_i>0$.

Suppose that $\mf b$ is the closest vector in $\mathcal L$ to $\mf a$, given by

\[\mf b=\begin{pmatrix}e_1c_1\\e_2c_2\\\vdots\\e_{n-1}c_{n-1}\\B\log\prod_{i=1}^np_i^{e_i}\end{pmatrix}\]

In some sense $B$ controls how big $e_i$ gets. The bigger $B$ gets the bigger $e_i$ gets.

Suppose that $e_i\neq0$ for all $i$. If any of them is zero we can just repeat this analysis with a smaller lattice probably.

By Minkowski's lattice point theorem, we have

\[B\log\left(\frac{\prod_{i=1}^np_i^{e_i}}{N}\right)\prod_{i=1}^{n-1}e_ic_i\leq|\det\mc L|=B\log p_n\prod_{i=1}^{n-1}c_i\]

Let $\varepsilon=\prod_{i=1}^np_i^{e_i}-N$, then we have

\[\log p_n\geq\left(1+\frac\varepsilon N\right)\prod_{i=1}^{n-1}e_i+O\left(\frac{\varepsilon^2}{N^2}\right)\]

Which gives us

\[\varepsilon\lessapprox N\left(\frac{\log p_n}{\prod_{i=1}^{n-1}e_i}-1\right)\]

and if we assume $e_i$ is somewhat random in a small range we immediately see that $\varepsilon\approx O(N)$, which tells us we need roughtly $O\left(\frac{N}{\log N^n}\right)$ lattices to obtain a fac-relation, which is pretty trash.

//todo: approx $e_i$ with $B$ but honestly looks quite bad lol

%\begin{thebibliography}{99} % alternatively can use bibtex
%    \bibitem[a]{b} c
%\end{thebibliography}

\end{document}
